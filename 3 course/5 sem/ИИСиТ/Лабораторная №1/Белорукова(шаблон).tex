\documentclass[a4paper,12pt]{article}
\usepackage[T2A]{fontenc}			
\usepackage[utf8]{inputenc}		
\usepackage[english,russian]{babel}

\title{Основы работы в \LaTeX}
\author{Белорукова Елизавета,1 подгруппа(1)}
\date{\today}

\begin{document}

\maketitle
\newpage
\section{Издательская система}

\textbf{Компьютерные издательские системы} -- это комплекс аппаратных и программных средств, предназначенных для компьютерного набора, верстки и издания текстовых и иллюстративных материалов. \\
Главным отличием настольных издательских систем от текстовых редакторов является то, что они предназначены, в первую очередь, для оформления документов, а не для ввода и редактирования. Процесс верстки состоит в оформлении текста и задании условий взаимного расположения текста и иллюстраций. Целью верстки является создание оригинал-макета, пригодного для размножения документа полиграфическим способом.

\subsection{Издательская система \TeX} 

\textbf{\TeX} -- система компьютерной вёрстки, разработанная американским профессором информатики Дональдом Кнутом в целях создания компьютерной типографии. В неё входят средства для секционирования документов, для работы с перекрёстными ссылками. \\
Многие считают \TeX лучшим способом для набора сложных математических формул. В частности, благодаря этим возможностям, \TeX популярен в академических кругах, особенно среди математиков и физиков.

\subsection{Дональд Кнут} 

\textbf{Дональд Эрвин Кнут} -- американский учёный в области информатики, эмерит-профессор Стэнфордского университета и нескольких других университетов в разных странах, в том числе Санкт-Петербургского, преподаватель и идеолог программирования, автор 19 монографий (в том числе ряда классических книг по программированию) и более 160 статей, разработчик нескольких известных программных технологий. Автор всемирно известной серии книг, посвящённой основным алгоритмам и методам вычислительной математики, а также создатель настольных издательских систем \TeX и METAFONT, предназначенных для набора и вёрстки книг научно-технической тематики (в первую очередь -- физико-математических).

\subsection{Издательская система \LaTeX} 

\textbf{\LaTeX} -- наиболее популярный набор макрорасширений (или макропакет) системы компьютерной вёрстки \TeX, который облегчает набор сложных документов.

Главная идея \LaTeX состоит в том, что авторы должны думать о содержании, о том, что они пишут, не беспокоясь о конечном визуальном облике (печатный вариант, текст на экране монитора или что-то другое). Готовя свой документ, автор указывает логическую структуру текста (разбивая его на главы, разделы, таблицы, изображения), а \LaTeX решает вопросы его отображения. Так содержание отделяется от оформления. Оформление при этом или определяется заранее (стандартное), или разрабатывается для конкретного документа.

Это похоже на стили оформления, которые используются в текстовых процессорах, или на использование стилевых таблиц в HTML.

\subsection{Лесли Лэмпорт} 

\textbf{Лесли Лэмпорт} -- американский учёный в области информатики, первый лауреат премии Дейкстры. Разработчик \LaTeX — популярного набора макрорасширений системы компьютерной вёрстки \TeX, исследователь теории распределённых систем, темпоральной логики и вопросов синхронизации процессов во взаимодействующих системах. Лауреат Премии Тьюринга 2013 года.

\newpage
\section{Основные правила создания текстового документа}

\end{document}
