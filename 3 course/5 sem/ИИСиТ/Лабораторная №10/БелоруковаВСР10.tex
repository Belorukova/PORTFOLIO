% "Тема 10. ВСР"

\documentclass[a4paper,12pt]{article} % тип документа

% report, book

%  Русский язык

\usepackage[T2A]{fontenc}			% кодировка
\usepackage[utf8]{inputenc}			% кодировка исходного текста
\usepackage[english,russian]{babel}	% локализация и переносы


% Математика
\usepackage{amsmath,amsfonts,amssymb,amsthm,mathtools} 


\usepackage{wasysym}

\usepackage{hyperref}

%Заговолок
\author{Белорукова Елизавета, ИВТ, 3 курс, 1 подгруппа}
\title{ВСР.}
\date{\today}


\begin{document} % начало документа
\maketitle
\newpage

\section{Аннотированный список онлайн сервисов \LaTeX}




\begin{itemize}

\item \large \textbf{\href{https://www.overleaf.com/}{Оverleaf}}

\normalsize Overleaf (ранее WriteLaTeX) - это интерактивный инструмент для совместного написания и публикации \LaTeX и Rich Text, который значительно ускоряет и облегчает весь процесс написания, редактирования и публикации научных документов.

\item \large \textbf{\href{https://sourceforge.net/projects/firetex/}{FireTeX}}

\normalsize FireTeX представляет собой редактор и компилятор LaTeX, реализованный в виде плагина к браузеру.\\
Возможности FireTeX:\\
Сохранение документа в форматах \LaTeX (.tex) и PDF.\\
Экспорт результатов в HTML, PDF и PNG.\\
Обмен информацией о результатах в формате HTML.\\
Подсветка синтаксиса \LaTeX.\\
Поиск и замена.\\
Кнопка для открытия файла \LaTeX.\\
Кнопка для экспорта кода \LaTeX.\\
Удобные кнопки для презентации в режиме слайд-шоу в формате HTML.\\
В настоящее время работает на браузерах, основанных на движке Blink: Google Chrome и Opera.\\


\item \large \textbf{\href{https://ru.sharelatex.com/}{ShareLaTeX}}

\normalsize Нет никаких трудностей с установкой, поэтому вы можете начать использовать \LaTeX сейчас, даже если никогда ранее не пробовали. ShareLaTeX доступен с полным и готовым к работе окружением \LaTeX, работающим на наших серверах.

С ShareLaTeX Вы получаете одинаковые настройки \LaTeX в любом месте. Работая с коллегами или учениками в ShareLaTeX, Вы никогда не столкнетесь с конфликтом версий или пакетов.

Также поддерживается большинство возможностей \LaTeX, включая вставку изображений, библиографии, формулы и многое другое.

\item \large \textbf{\href{https://papeeria.com/}{Papeeria}}

\normalsize Онлайн сервис Papeeria включает в себя бесплатный совместный редактор \LaTeX и конструктор сюжетов. Результатом компиляции обычно является PDF документ, и если он получился, то он будет показан рядом с вашим исходным текстом. Также Papeeria имеет большую галерею шаблонов и неограниченные проекты, которые можно выполнять с соавторами. Основной функционал сервиса бесплатный.

\item \large \textbf{\href{https://www.xm1math.net/texmaker/}{TEXMAKER}}

Если вы фанат GNOME и используете такие дистрибутивы Linux, как Ubuntu или Debian, вам следует рассмотреть TeXmaker. Это один из самых профессиональных и известных редакторов LaTeX, доступных на Linux. Он имеет ряд особенностей, разработанных для того, чтобы сделать дизайн \LaTeX более приятным для пользователей. К ним относятся такие функции, как проверка орфографии и триггеры клавиатуры, а также поддержка восемнадцати языков. Если вы хотите перейти с вашего Linux PC на Macbook, то сможете это сделать, благодаря кросс-платформенной поддержке. Вы также можете просматривать документы в процессе работы, благодаря встроенной в программе просмотр PDF-документов.

\item \large \textbf{\href{https://www.lyx.org/}{LyX}}

\normalsize Ищете инструмент редактирования \LaTex, который очень похож на ваш любимый текстовый редактор? LyX может быть вашим выбором. Простой и привычный дизайн облегчает создание документов LaTeX, при этом структура вашего документа имеет первостепенное значение. \\
Вы устанавливаете правила работы с документами благодаря инструментам автоформатирования LyX. Если у вас есть регулярные триггеры (например, определенные ключевые слова или заголовки), вы можете настроить LyX на автоматическое форматирование их для вас. В комплект также входит полезное руководство для новичков и подробное руководство, которое поможет вам быстро освоиться.\\
Это один из старейших и самых старых поддерживаемых редакторов \LaTeX, который используется с 1995 года.\\

\item \large \textbf{\href{https://kile.sourceforge.io/}{Kile}}

\normalsize Нет ничего лучше простой среды разработки, в которой все, от вашего кода до выходных данных, можно увидеть в одном окне. С Kile вы получите именно это.\\
Все разработано для того, чтобы упростить процесс создания документов \LaTeX, но это далеко не все. Вы можете создавать шаблоны документов, легко вставлять изображения, автоматически выполнять команды \LaTeX для экономии времени, а также интегрировать свой документ в BibTeX, основной инструмент для научных справок.\\
Вы также можете легко просматривать документы в формате PDF.\\

\end{itemize}

\end{document} % конец документа