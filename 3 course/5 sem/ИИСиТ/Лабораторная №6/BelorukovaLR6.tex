% "Лабораторная работа 6. № 1"

\documentclass[a4paper,12pt]{article} 





\usepackage[T2A]{fontenc}			
\usepackage[utf8]{inputenc}			
\usepackage[english,russian]{babel}	



\usepackage{amsmath,amsfonts,amssymb,amsthm,mathtools} 


\usepackage{wasysym}
\usepackage{indentfirst} 
\usepackage{hyperref}

%Заговолок
\author{Белорукова Елизавета, ИВТ, 3 курс, 1 подгруппа}
\title{Работа с текстом в \LaTeX{}}
\date{28.11.2019}


\begin{document} 

\maketitle
\newpage
\begin{center}
\section{Для чего предназначена издательская система \LaTeX{}}
\end{center}

\large
\begin{flushright}
\LaTeX{} -- система верстки, ориентированная на производство научных математических документов высокого типографического качества. Система также вполне подходит для производства других видов документов, от простых писем до полностью сверстанных книг. \LaTeX{} использует \href{https://ru.wikipedia.org/wiki/TeX}{\TeX{}} в качестве своего механизма верстки.
\end{flushright}

\section{В каких случаях рационально её использовать}

Главными пользователями \LaTeX{} являются учёные.

Если вам необходимо красиво оформить текст \LaTeX{} определенно отличный выбор.

Вы можете рассмотреть \LaTeX{} как вариант самостоятельной вёрстки книги для издательства. Академические издатели ожидают подобного от учёных и также могут примириться с таким вариантом от исследователей в сфере гуманитарных наук. Это то, для чего Кнут\footnote{Дональд Эрвин Кнут -- создатель настольных издательских систем \TeX и METAFONT} изначально изобрёл \TeX{}.

\section{Какие преимущества имеет работа в этот системе}

\LaTeX{} позволяет автоматизировать многие задачи набора текста и подготовки статей, включая набор текста на нескольких языках, нумерацию разделов и формул, перекрёстные ссылки, размещение иллюстраций и таблиц на странице, ведение библиографии и др. 

Большое количество макропакетов, позволяющих сделать все, о чем можно только подумать.

Гибкость. Все, что вы можете представить себе, будет исполнено с минимальными затратами труда (только профессионалом).

Поддержка любых языков в рамках одного документа.

Строгий подход к оформлению удерживает пользователя в рамках полиграфических приличий.

Более сложные элементы текста, такие как сноски, библиография, оглавление, список таблиц и т. п., а также простые рисунки могут быть выполнены без особых трудностей.

\section{Какие сложности могут возникнуть при работе в этот системе}
\begin{flushright}
\begin{itemize}
\item У свободного и бесплатного программного обеспечения есть большая склонность к трудностям в установке и работе. \TeX{} и \LaTeX{} не исключение.

\item Документы \LaTeX{} очень сложно читать до вёрстки, что делает неэффективным как написание текстов, так и редактирование. Документы \LaTeX{} могут превращаться в прекрасные читаемые PDF после набора и вёрстки, но метод их редактирования не оптимальный.
\end{itemize}
\end{flushright}
\section{Какие недостатки отмечают пользователи при работе с этой системой}

\begin{enumerate}
\item Требуется владеть навыками работы в редакторе текста.

\item Ориентирован на многозадачную среду.

\item Требует знания элементарных основ полиграфии.

\item Создание новых стилей оформление - дело сложное и под силу лишь профессионалам. Обычный пользователь, как правило, с такой задачей не справится.
\end{enumerate}

\end{document} 