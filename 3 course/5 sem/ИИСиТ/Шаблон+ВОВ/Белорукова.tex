\documentclass[a4paper,12pt]{article}
\usepackage[T2A]{fontenc}			
\usepackage[utf8]{inputenc}		
\usepackage[english,russian]{babel}

\title{Родственники - участники Великой Отечественной войны}
\author{Белорукова Елизавета, ИВТ, 3 курс, 1 подгруппа}
\date{\today}

\begin{document}

\maketitle
\newpage
\textbf{Вели́кая Оте́чественная война́ (22 июня 1941 года — 9 мая 1945 года) — война Союза Советских Социалистических Республик против вторгшихся на советскую территорию нацистской Германии и её европейских союзников (Венгрии, Италии, Румынии, Словакии, Финляндии, Хорватии). Важнейшая составная часть Второй мировой войны, завершившаяся победой Красной Армии и безоговорочной капитуляцией вооружённых сил Германии. В западных странах именуется Восточным фронтом[4], в Германии — также Немецко-Советской войной.

Великая Отечественная война затронула каждую семью,многие понесли потери. Война затронула и нашу семью, я расскажу вам о своей прабабушке: \\
\textit{Савина Нина петровна }.\\[10pt]

\section*{Савина Нина Петровна}
Моя прабабушка.\\[3pt]
\textbf{Дата рождения}: 06.05.1928, п. Оредеж, Лужского р-на, Ленинградская обл.\\[3pt]


\textit{В 1939 году} когда ей было 11 она проживала в поселке Оредеж со своей семьей. Их загнали в церковь и хотели сжечь всем поселком, но ее спас немец, который помог ей отправиться в Австрию,где она несколько лет работала служанкой в доме у богатой женщины. 
\textit{Вернулась домой живой}  \textit{осенью 1945 году.}



\end{document}
