% "Тема 8. ИСР"

\documentclass[a4paper,12pt]{article} % тип документа

% report, book

%  Русский язык

\usepackage[T2A]{fontenc}			% кодировка
\usepackage[utf8]{inputenc}			% кодировка исходного текста
\usepackage[english,russian]{babel}	% локализация и переносы


% Математика
\usepackage{amsmath,amsfonts,amssymb,amsthm,mathtools} 


\usepackage{wasysym}

%Заговолок
\author{Белорукова Елизавета, ИВТ, 3 курс, 1 подгруппа}
\title{Тема 8. ИСР}
\date{\today}

\begin{document}

\maketitle
\newpage
\section{Команды для набора матриц в \LaTeX{}}

\begin{tabular}{| l | l |}
\hline
\textbf{Назначение команды} & \textbf{Вид (написание) команды} \\
\hline
Матрица без скобок & \verb|\begin{matrix}|\\
  & \verb|... & ...\\|\\
  & \verb|... & ...\\|\\
  & \verb|\end{matrix}|\\
Матрица в квадратных скобках & \verb|\begin{bmatrix}|\\
  & \verb|... & ...\\|\\
  & \verb|... & ...\\|\\
  & \verb|\end{bmatrix}|\\
Матрица в фигурных скобках & \verb|\begin{Bmatrix}|\\
  & \verb|... & ...\\|\\
  & \verb|... & ...\\|\\
  & \verb|\end{Bmatrix}|\\
Матрица в круглых скобках & \verb|\begin{pmatrix}|\\
  & \verb|... & ...\\|\\
  & \verb|... & ...\\|\\
  & \verb|\end{pmatrix}|\\
Одинарный модуль & \verb|\begin{vmatrix}|\\
  & \verb|... & ...\\|\\
  & \verb|... & ...\\|\\
  & \verb|\end{vmatrix}|\\
Двойной модуль (норма) & \verb|\begin{Vmatrix}|\\
  & \verb|... & ...\\|\\
  & \verb|... & ...\\|\\
  & \verb|\end{Vmatrix}|\\
\hline
\end{tabular}

\end{document}
