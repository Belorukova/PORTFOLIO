
\documentclass[a4paper,12pt]{article} % тип документа

% report, book

%  Русский язык

\usepackage[T2A]{fontenc}			% кодировка
\usepackage[utf8]{inputenc}			% кодировка исходного текста
\usepackage[english,russian]{babel}	% локализация и переносы
\usepackage{hyperref}               % гиперссылки


% Математика
\usepackage{amsmath,amsfonts,amssymb,amsthm,mathtools} 


\usepackage{wasysym}

%Заговолок
\author{Белорукова Елизавета, ИВТ, 3 курс, 1 подгруппа}
\title{\textbf{Издательская система Adobe FrameMaker}}
\date{2019 г.}


\begin{document} % начало документа

\maketitle
\newpage

\begin{center}
\huge \textbf{Оглавление}
\end{center}
\Large
Введение ...............................................................\Large3\\
Основная часть ....................................................\Large4

\large
Adobe FrameMaker .....................................................\Large4

\large 
Особенности издательской системы Adobe FrameMaker ...........\Large5

\large 
Сфера применения Adobe FrameMaker ......................................\Large6\\
\Large Заключение ..........................................................\Large8\
\newpage

\begin{center}
\huge \textbf{Введение}
\end{center}

\normalsize

Компьютерные издательские системы -- это комплекс аппаратных и программных средств, предназначенных для компьютерного набора, верстки и издания текстовых и иллюстративных материалов. 

Главным отличием настольных издательских систем от текстовых редакторов является то, что они предназначены, в первую очередь, для оформления документов, а не для ввода и редактирования. Процесс верстки состоит в оформлении текста и задании условий взаимного расположения текста и иллюстраций. Целью верстки является создание оригинал-макета, пригодного для размножения документа полиграфическим способом.

Настольные компьютерные издательские системы приобрели широкую популярность в различных сферах производства, бизнеса, науки, культуры и образования.

Издательское дело становится актуальным практически для любой организации. Выпуск информационных бюллетеней, рекламных проспектов, собственных малотиражных газет и даже книг теперь становится необходимым атрибутом информационного обеспечения современных учреждений. Пожалуй, из всех новых информационных технологий, компьютерное издательство является наиболее массовой и практически легко внедряемой.

\newpage

\begin{center}
\huge \textbf{Основная часть}\\[1cm]

\Large \textbf{Adobe FrameMaker}
\end{center}

\normalsize

        Adobe FrameMaker — настольная издательская система, основанная на структурированной разметке документа (см. SGML, HTML, XML), в отличие от систем, основанных на подобном  верстаку графическом  интерфейсе (GUI).

        Компания Adobe знаменита на весь мир своими программными продуктами. Однако такой комплекс как FrameMaker известен лишь малому кругу пользователей. Между тем, данному продукту уже более 30 лет. Это специализированное программное обеспечение, предназначенное специально для специалистов в области издательского дела.

         С помощью Adobe FrameMaker можно создавать документы любой сложности и работать с большим количеством иллюстраций. Комплекс поддерживает различные шрифты, добавление ссылок и прочие функции, характерные для текстовых редакторов. Ключевой особенностью, за который полюбили Adobe FrameMaker, является возможность увеличения масштаба при сохранении всех пропорций. Это избавляет от необходимости регулярно проводить верстку, как это происходит с текстовым редактором Word.

\newpage

\begin{center}
\Large \textbf{Особенности издательской системы Adobe FrameMaker}
\end{center}

\normalsize

        Adobe FrameMaker поддерживает множество форматов, что позволяет адаптировать проект под различные виды платформ. В распоряжении пользователя есть шаблоны, широкий набор инструментов, панель для написания скриптов. Современные версии Adobe FrameMaker позволяют конфигурировать интерфейс таким образом, чтобы облегчить работу.     Поддерживается несколько разных языков.
         Важным преимуществом продукта является поддержка QuarkXPress и методов обработки текста аналогично Word. Благодаря этому можно привязывать любые объекты к любым участкам текста. Автоматическая верстка налажена таким образом, чтобы избавить пользователя от многих проблем. Более того, FrameMaker позволяет легко работать с таблицами. При этом важным преимуществом перед Word является поддержка стилей таблиц. 
         Adobe FrameMaker также предусматривает опцию создания перекрестных ссылок с возможностью сослаться на любую часть текста. Для решения проблем с Word многим пользователям приходится применять макросы. FrameMaker решает эту проблему, что позволяет быстро проводить верстку в словарях. Достаточно добавить ссылку из колонтитула на первое слово страницы.
        Наконец, FrameMaker удобен в плане нумерации. Для этого в комплексе задействуются формулы, а это позволяет создавать вложенную нумерацию и добавить к ней самую разную информацию. А еще у FrameMaker есть эталонные страницы, на которых расположены произвольные проименованные объекты. Вызвать любой объект можно, указав имя, после чего он появится в нужном участке текста.
          В итоге FrameMaker – один из лучших вариантов для верстки книг и журналов. Программа заслужила широкое признание издательств благодаря хорошей автоматизации, возможности использования опций для создания шаблонов и экономии времени на стандартных процедурах.


Поддерживает платформы: 
\begin{itemize}
\item Microsoft Word 2000, 2003, 2007
\item Adobe Acrobat 8, 9, X
\item Adobe Captivate 4, 5
\item Adobe Illustrator CS5
\end{itemize}

\newpage

\begin{center}
\Large \textbf{Сфера применения Adobe FrameMaker}
\end{center}

\normalsize

 Пакет FrameMaker присутствует на рынке настольных издательских систем уже более 20 лет. За это время он получил устойчивую репутацию лучшего программного средства для подготовки сложных многостраничных публикаций. Объективная статистика показывает, что большая часть таких изданий, как технические монографии, учебники по естественно-научным дисциплинам, словари, энциклопедии, справочники и т. д., готовятся к печати в среде FrameMaker. Какие же особенности с делали пакет лидером в этом секторе издательского бизнеса? FrameMaker обладает очень устойчивым движком. Много лет интенсивной эксплуатации позволили выявить и устранить большую часть программных ошибок. Чистота кода сочетается с высокой реактивностью пакета. Сложные директивы форматирования, документа, FrameMaker отрабатывает «на лету», позволяя верстальщику оперативно оценить принятое решение, требующие масштабной переверстки. В состав пакета включены специальные средства разработки объемных публикаций. Аппарат создания автоматических нумераций, инструменты для работы генерации порожденных файлов и множество других технических средств позволяют сверстать макет любой сложности. Программа ориентирована на единообразное форматирование и воспроизводство повторяемых оформительских решений. Это достигается продуманным применением шаблонов и стилей Шаблоны FrameMaker представляют собой контейнеры оформительских данных Информационной емкостью они с большими таблицами, редактор математических формул, команды намного превосходят шаблоны других настольных издательских систем и способны хранить большую часть оформительских атрибутов публикации Шаблоны FrameMaker позволяют накапливать оформительские данные и обмениваться  между проектами. Мощное средство автоматической верстки, которое существенно упрощает переиздание публикациями. 
         FrameMaker это одна из немногих программ, полноценно поддерживающих работу такую технологию разработки, которая наделяет документ способностью существовать в различных информационных средах: на диске, в корпоративной локальной сети, на сайте во с единым источником (single sourcing). Эта главным возможность реализована в пакете очень экономными средствами образом при помощи условных стилей и конверторов в основные печатные и электронные форматы (НТML, XML, PDF и др.). Условные стили позволяют менять состав и вид документа, а конверторы обеспечивают вывод версий на различные носители. 

        Adobe FrameMaker представляет собой мощное решение для создания авторского контента и публикации техническими специалистами.           Воспользуйтесь преимуществами интуитивного пользовательского интерфейса, универсальных рабочих процессов и сред разработки авторского контента на основе шаблонов для упрощения публикации контента и соблюдения организационных требований к согласованности и фирменному оформлению.

         Вы сможете легко создавать неструктурированный, структурированный, и XML/DITA контент с помощью передовых шаблонов и инструментов. Вы можете объединять несколько типов контента, использовать самые сложные скрипты и расширенные графические возможности, улучшая тем самым удобство пользования.


\newpage

\begin{center}
\huge \textbf{Заключение}

\end{center}

\normalsize

           Повысьте эффективность работы благодаря соблюдению всех стандартов, предустановленных инструментов и шаблонов, упрощающих процесс авторинга. Используйте команды Auto Spell Check, Highlight Support, Find and Replace для повышения скорости и эффективности выполнения задач.

\newpage


\end{document}
