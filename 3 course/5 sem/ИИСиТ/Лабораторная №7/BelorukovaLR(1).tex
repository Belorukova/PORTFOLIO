% "Лабораторная работа 7. Задание 1"

\documentclass[a4paper,12pt]{article} % тип документа

% report, book

%  Русский язык

\usepackage[T2A]{fontenc}			% кодировка
\usepackage[utf8]{inputenc}			% кодировка исходного текста
\usepackage[english,russian]{babel}	% локализация и переносы


% Математика
\usepackage{amsmath,amsfonts,amssymb,amsthm,mathtools} 


\usepackage{wasysym}

\usepackage{hyperref}

%Заговолок
\author{Белорукова Елизавета, ИВТ, 3 курс, 1 подгруппа}
\title{Особенности технологии создания текста с формулами. Задание 1}
\date{\today}


\begin{document} % начало документа
\maketitle
\newpage

\begin{enumerate}

\item \large \textbf{Как заставить \LaTeX нумеровать уравнения?}

\normalsize Если вы хотите, чтобы \LaTeX нумеровал ваши уравнения, используйте окружение equation.

\item \large \textbf{Чема математический режим отличается от текстового режима?}

\normalsize Большинство пробелов и возвратов картеки не принимаются во внимание. Пустые строчки недопустимы. Каждая буква считается именем переменной, и верстается в этом качестве.

\item \large \textbf{В чём различие ввода греческих букв?}

\normalsize Строчные греческие буквы вводятся как \verb|\alpha|, \verb|\beta|, \verb|\gamma| и т.д. Прописные буквы вводятся как \verb|\Gamma|, \verb|\Delta| и т.д.

\item \large \textbf{Как вводятся верхние и нижние индексы?}

\normalsize Для ввода верхнего индекса используется символ \verb|^|, а для ввода нижнего индекса используется символ \verb|_|.

\item \large \textbf{Как вводить число в корне?}

\normalsize Квадратный корень вводится как \verb|\sqrt|, корень $n$-ной степени печатается при помощи \verb|\sqrt[n]|. Размер знака корня выбирается \LaTeX автоматически. Если нужен один только знак, используйте \verb|\surd|.

\item \large \textbf{Как создать горизонтальные фигурные скобки для выражения?}

\normalsize Команда \verb|\overbrace| создает длинную горизонтальную фигурную скобку на выражением, а команда \verb|\enderbrace| создает горизонтальную фигурную скобку под выражением.

\item \large \textbf{Как указать вектор над переменными?}

\normalsize Векторы указываются с помощью команды \verb|\vec|. Для обозначения вектора от A до B полезны две команды \verb|\overrightarrow| и \verb|\overleftarrow|.

\item \large \textbf{Как ввести знак умножения в виде точки?}

\normalsize Обычно знак точки, обозначающий умножения, явно не набирается. Однако, иногда он полезен, чтобы помочь читателю сгрупировать формулу. Ипользуйте для этого команду \verb|\cdot|.

\item \large \textbf{Как сверстать двухъярусную дробь?}

\normalsize Двухъярусная дробь верстается командой \verb|\frac{}{}|. Часто предпочтительнее её форма с косой чертой \verb|1/2|, потому что она смотрится лучше при небольшом количестве дробного материала.

\item \large \textbf{Как ввести в формулу три точки?}

\normalsize Чтобы ввести в формулу три точки, если несколько команд. \verb|\ldots| верстает точки на базовой линии, \verb|\cdots| -- центрированные. Кроме того, существует команда \verb|\vdots| для вертикальных и \verb|\ddots| для диагональных точек.

\end{enumerate}

\end{document} % конец документа